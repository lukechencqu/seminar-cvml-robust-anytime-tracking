\input{header}

\newcommand{\backupbegin}{
   \newcounter{finalframe}
   \setcounter{finalframe}{\value{framenumber}}
}
\newcommand{\backupend}{
   \setcounter{framenumber}{\value{finalframe}}
}

% define vector/matrix helpers
\newcommand*\colvec[3][]{
    \begin{pmatrix}\ifx\relax#1\relax\else#1\\\fi#2\\#3\end{pmatrix}
}
\newcommand{\mattwo}[4]{\begin{pmatrix} #1 & #2 \\ #3 & #4\end{pmatrix}}


\title[Combining 3D Shape, Color, and Motion for Robust Anytime Tracking]{Combining 3D Shape, Color, and Motion\\ for Robust Anytime Tracking\\ \small{Paper by Held, Levinson, Thrun, and Savarese~\cite{paper}}}
\author[Zwilling]{%
  Frederik Zwilling\\
  \bigskip
  {\scriptsize Seminar Current Topics in Computer Vision and Machine Learning\\ RWTH Aachen University}
}

\date[\today @ Seminar CVML]{\today -- Seminar Current Topics in Computer Vision and Machine Learning}

\begin{document}

\frame[plain]{\titlepage}
\addtocounter{framenumber}{-1}

\begin{frame}
  \frametitle{Agenda}
  \tableofcontents[hideallsubsections]
\end{frame}

\section{Motivation}

\begin{frame}
  \frametitle{Tracking for Autonomous Cars}
  
  \begin{columns}
  \begin{column}{0.6\textwidth}
  \begin{description}[]
  \item[Chances] \hfill \\
  \begin{itemize}
  \item Free use of driving time
  \item Help disabled persons
  \item Computers do not get\\tired or drunk
  \item Faster reaction time
  \item[$\Rightarrow$] Safe 26k lifes per year in EU
  \end{itemize}
  \end{description}
  \end{column}
  \begin{column}{0.4\textwidth}
  \includegraphics[width=\textwidth]{images/auto.jpg}
  \end{column}
  \end{columns}
  \pause
  \bigskip
  \begin{columns}
  \begin{column}{0.6\textwidth}
  \includegraphics[height=0.35\textheight]{images/highway}
  \hspace{0.01cm}
  \includegraphics[height=0.35\textheight]{images/cycle}
  \hspace{0.01cm}
  \includegraphics[height=0.35\textheight]{images/pedestrian}
  \end{column}
  \begin{column}{0.5\textwidth}
  \begin{description}[]
  \item[Challenges] \hfill \\
  \begin{itemize}
  \item Precise tracking
  \item Robustness
  \item Occlusion
  \item Real time
  \end{itemize}
  \end{description}
  \end{column}
  \end{columns}
\end{frame}

\begin{frame}
  \frametitle{Tracking for Autonomous Cars}
  \begin{description}[]
  \item[Subtasks of Tracking] \hfill \\
  \begin{itemize}
  \item Segment sensor data into objects
  \item Associate objects in successive frames
  \item Position and velocity estimation
  \item Object and trajectory classification
  \end{itemize}
  \end{description}
  \pause
  \begin{block}{Topic of this presentation}
    Position and velocity estimation
  \end{block}
\end{frame}

\begin{frame}
  \frametitle{Given Sensor Data}

  \begin{columns}
  \begin{column}{0.6\textwidth}
  \begin{description}[]
  \item[Sensor] \hfill \\
  \begin{itemize}
  \item Dense 3D laser sensor
  \item Generates point cloud
  \item Additional panorama image
  \item Similar to stereo cameras\\
        but more precise and expensive
  \end{itemize}
  \pause
  \item[Given for us] \hfill \\
  \begin{itemize}
  \item Point clouds of detected objects
  \item Association between frames already done
  \end{itemize}
  \end{description}
  \end{column}
  \onslide<1->
  \begin{column}{0.4\textwidth}
  \includegraphics[width=\textwidth]{../img/lidar}\\
\bigskip
  \includegraphics[width=\textwidth]{../img/lidar-data}
  \end{column}
  \end{columns}
\end{frame}

\begin{frame}
  \frametitle{Teaser Paper Ideas}
  \begin{description}[]
  \item[How to find a precise alignment?] \hfill \\
  \begin{itemize}
  \item Utilize whole object shape
  \item Additional cues from color
  \item Use motion model
  \end{itemize}
  \pause
  \item[How to search the state space fast?] \hfill \\
  \begin{itemize}
  \item Histogram with coarse initial resolution
  \item Refine resolution important areas
  \item Consider resolution in the probabilistic model
  \end{itemize}
  \end{description}
\end{frame}


\section{Baseline Methods}
\begin{frame}
  \frametitle{Baseline Methods}
  \begin{tikzpicture}[thick, every node/.style={font=\footnotesize}]
    \only<1>{
    \node (program) [outer sep=0,inner sep=0,anchor=south west]
    at (0,0)
    {
    \includegraphics[width=\textwidth,height=0.25\textwidth]{images/point-clouds}};}
    \only<2>{
    \node (program) [outer sep=0,inner sep=0,anchor=south west]
    at (0,0)
    {
    \includegraphics[width=\textwidth,height=0.25\textwidth]{images/centroid}};}
    \only<3>{
    \node (program) [outer sep=0,inner sep=0,anchor=south west]
    at (0,0)
    {
    \includegraphics[width=\textwidth,height=0.25\textwidth]{images/centroid-alignment}};}
    \only<4->{
    \node (program) [outer sep=0,inner sep=0,anchor=south west]
    at (0,0)
    {
    \includegraphics[width=\textwidth,height=0.25\textwidth]{images/kalman-error}};}
    \node (descx) at (10.6,-0.25)
    {\scriptsize \cite{held-website}};
    \onslide<1->
  \end{tikzpicture}
  
  \begin{description}[]
  \item[Kalman Filter] \hfill \\
  \begin{itemize}
  \onslide<2->{\item Aligns centroids}
  \onslide<4->{\item Problems with occlusion}
  \onslide<5->{\item Robustness through motion model}
  \onslide<6->{\item Very fast}
  \end{itemize}
  \end{description}
\end{frame}

\begin{frame}
  \frametitle{Baseline Methods}
  
  \begin{description}[]
  \item[Iterative Clostest Point (ICP)] \hfill \\
  \begin{itemize}
  \item Iterative hill climbing approach
  \item Minimizes quadratic distance of closest points
  \item Uses whole point cloud
  \pause
  \item Depends on good initialization
  \item Problem: local optima\\
  \begin{tikzpicture}[thick, every node/.style={font=\footnotesize}]
    \only<2>{
    \node (program) [outer sep=0,inner sep=0,anchor=south west]
    at (0,0)
    {
    \includegraphics[height=0.25\textwidth]{images/icp-init}};}
    \only<3>{
    \node (program) [outer sep=0,inner sep=0,anchor=south west]
    at (0,0)
    {
    \includegraphics[height=0.25\textwidth]{images/icp-1}};}
    \only<4>{
    \node (program) [outer sep=0,inner sep=0,anchor=south west]
    at (0,0)
    {
    \includegraphics[height=0.25\textwidth]{images/icp-2}};}
    \only<5->{
    \node (program) [outer sep=0,inner sep=0,anchor=south west]
    at (0,0)
    {
    \includegraphics[height=0.25\textwidth]{images/icp-3}};}
    \node (descx) at (5.7,-0.2)
    {\scriptsize \cite{held-website}};
    \onslide<1->
  \end{tikzpicture}
  \pause
  \pause
  \pause
  \item No motion model
  \end{itemize}
  \end{description}
\end{frame}


\section{Probabilistic Model}

\begin{frame}
  \frametitle{Probabilistic Model}      
  
\end{frame}

\begin{frame}
  \frametitle{Probabilistic Model}
  \begin{itemize}
  \item Dynamic Bayesian Network
  \item Surface points - Measurement points
  \item Derivation of measurement model
  \item Color model
  \item Motion model
  \end{itemize}
\end{frame}

\section{Searching the State Space}
\begin{frame}
  \frametitle{Searching the State Space}
  \begin{itemize}
  \item Dynamic histogram
  \item Refinement step
  \item Annealing
  \end{itemize}
\end{frame}

\section{Evaluation}
\begin{frame}
  \frametitle{Evaluation}
  \begin{enumerate}
  \item Videos from website
  \item Relative reference frame approach
  \item Comparison with Kalman and ICP variants
  \item Comparison ADH - densly sampling
  \item Model Crispness approach
  \item Provided Code, experiments
  \end{enumerate}
\end{frame}

\section{Conclusion}
\begin{frame}
  \frametitle{Conclusion}
  \begin{itemize}
  \item Catch Phrase
  \item Main Points
  \end{itemize}
\end{frame}



\backupbegin

\begin{frame}[allowframebreaks]
  \frametitle{References}
  %% \nocite{*}
  \bibliographystyle{splncs}
  \bibliography{references}
\end{frame}

\begin{frame}
  \frametitle{Missing stuff}
  \begin{itemize}
  \item related work: alternative sensors, grid-based methods
  \end{itemize}
\end{frame}

\backupend

\end{document}
